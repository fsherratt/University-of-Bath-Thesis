\chapter*{Abstract} % This is a maximum of 1 page and 300 words!
Lower limb amputees struggle with an impaired gait that traditional prostheses' performance cannot entirely correct. The development of powered prosthetic devices aims to solve this by replacing the power generating muscles. Powered prostheses will only be successful if the device is effectively controlled.

A human gait involves multiple control modes for different activities; for a non-amputee, transitions between these modes are seamless and natural. In a powered prosthetic device, these levels of control need replicating. The correct selection of gait mode based on an individual's intent is crucial. Machine Learning (ML) methods are a promising avenue for this. However, their applicability to amputees is under-researched; this is partly due to the difficulty in collecting amputee gait data. This research aims to investigate ML methods for Locomotion Mode Recognition in amputees while reducing the data requirements for its implementation.

An extensive public data set of gait data is collected using novel wireless Inertial Measurement Units (IMU) and a companion smartphone app for labelling activities in real-time. The gait data set is used to investigate the performance of an Long Short Term Memory (LMR) network for non-amputees. The analysis identifies that the model primarily classifies activity type based on data around early stance, a period with significant difference between individuals. The model also struggles to generalise to novel unseen users due to over fitting to the subjects' individual gait traits. Therefore personalisation is required. 

Subsequently, methods for personalisation are investigated. Transfer learning is identified as a promising research field. However, its application to IMU amputee gait data has not previously been demonstrated. A novel method for dividing continuous, unstructured and poorly distributed gait data is developed to investigate personalisation methods: this successfully improves classification performance and reduces data requirements in both amputees and non-amputees.
