\chapter{Introduction}
\label{chp:intro}

\section{Motivation}
% Copied from transfer report
Lower limb amputation affects a small but significant portion of the population with the number of people living with amputations predicted to continue rise. More than one million amputations occur globally that is one every 30 seconds.\cite{Asif2021}Limb loss often occurs due to traumatic injuries, certain diseases, and forced amputation due to surgery; increasingly as a result of vascular disease or diabetes\cite{Griffin2012}. Regardless of amputation aetiology, lower limb amputees require more energy to walk compared with their non-amputee peers\cite{Vllasolli2014}.

The loss of a lower limb dramatically impedes movement\cite{Gregg2014, Wong2021, Srisuwan2021}, reducing amputees quality of life and increasing the risk of further compensatory injuries. The use of a powered prosthesis effectively replace the lost limb, reduce energy expenditure during walking and rebalance gait\cite{Lin2014}. In order to effectively control the prosthetic user intent and activity must be known. This knowledge must be obtained in real-time using only information that can be gathered through local sensors.

For the able bodied transitioning between locomotion modes is a seamlessly process with both legs adapting gait method to the activity. In order for a leg prosthetic to feel truly natural its controller must be capable of the same seamless behaviour. The identification of appropriate locomotion mode is a vital component of this. 

Amputees have a wide range of gait abnormalities usually through compensatory mechanisms\cite{Tucker2015}. Accounting for this increases the need for a highly adaptable controller. The use of hard coded rules requires both a large human input but also reduces potential adaption. \acrfull{ml} has the potential to fully adapt to individual users by learning their personal behaviour.


%--------------------------------------------------------------------------------
\section{Hypothesis}
Controllers will need to be different for different locomotive activities - therefore a systems which can establish a accurate, reliable and timely determination of locomotive mode are essential.

Machine-learning techniques can be used to effectively implement such a system personalising to the unique needs of each amputee.


%--------------------------------------------------------------------------------
\section{Aims and Objectives}
The objective of this work is to develop a


%--------------------------------------------------------------------------------
\section{Contributions}
The main contributions of this works is a \hl{...}. The specific contributions to knowledge are
contained in each chapter are as follows.

\begin{itemize}
    \item In Chapter \ref{chp:background} a
    
    \item In Chapter \ref{chp:methods} a novel approach to large scale collection of labelled gait data is presented as well as a new data set for evaluation of \acrshort{har} classifiers.
    
    \item In Chapter \ref{chp:lstm-general} development of a LSTM HAR classifier and it's underlying behaviour
    
    \item In Chapter \ref{chp:personalisation} methods for applying transfer learning to 
    
    \item Finally in Chapter \ref{chp:amputee-data} demonstration of the use of effective use of personalising methods on a trans-tibial amputee to improve the performance of a classifier.
\end{itemize}

%--------------------------------------------------------------------------------
\section{Thesis Structure}
