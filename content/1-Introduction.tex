\chapter{Introduction}
\label{chp:intro}

\section{Motivation}
% Copied from transfer report
Lower limb amputation affects a small but significant portion of the population with the number of people living with amputations predicted to continue rise. Increasingly as a result of vascular disease or diabetes\cite{Griffin2012}.%amputation will be caused by diabetes. Diabetics are 10 times more likely to need a lower limb amputation\cite{whoStat2013} with over 80\% of all amputations in the UK as a result of vascular disease or diabetes\cite{Griffin2012}. 
The loss of a lower limb dramatically impedes movement\cite{Gregg2014}, reducing amputees quality of life and increasing the risk of further compensatory injuries. The use of a powered prothesis effectively replace the lost limb, reduce energy expenditure during walking and rebalance gait\cite{Lin2014}. In order to effectively control the prosthetic user intent and activity must be known. This knowledge must be obtained in real-time using only information that can be gathered through local sensors.

For the able bodied transitioning between locomotion modes is a seamlessly process with both legs adapting gait method to the activity. In order for a leg prosthetic to feel truly natural its controller must be capable of the same seamless behaviour. The identification of appropriate locomotion mode is a vital component of this. 

Amputees have a wide range of gait abnormalities usually through compensatory mechanisms\cite{Tucker2015}. Accounting for this increases the need for a highly adaptable controller. The use of hard coded rules requires both a large human input but also reduces potential adaption. Machine learning has the potential to fully adapt to individual users by learning their personal behaviour.

\section{Hypothesis}
Can machine-learning techniques be used to improve locomotion-mode selection in lower-limb prosthesis?


\section{Aims and Objectives}


\section{Contributions}