\chapter{Background}
Introduction to chapter

%---------------------------------------------%
\section{Bio-mechanics of Gait}
Overview of Gait 
%   - cyclic process - highly efficient
%   - key events and terminology [Swing, Stance, IC/HS, TO]
%   - picture of gait cycle

How is it measured (Cadence, stride length, toe clearance)

What causes it to vary, how does it vary (Note: for non-amputees these changes are automatic sub-conscious)
%   - Different activities (W, S, RA, RD, SA, SD)
%   - Gait can therefore be broken up into different locomotive modes that require different control loops

What are the differences in gait for amputees?
%   - Compensatory mechanisms
%   - Reduced power generating muscles - decreases gait efficiency
%   - Don't need two much detail will discuss more in chapter 6


%---------------------------------------------%
\section{Prostheses}
Section introduction

%(Is this part of motivation in intro?)
Described amputation and it's prevalence - why is research in this area relevant 

What is the aim of prosthetic devices

\subsection{History of Prosthesis}
History of prosthesis

Traditionally mechanically passive

What is current state of the art (Powered prosthesis)

\subsection{Control requirements} % Introduce the problem
Powered prosthesis require control to ensure they work in unison
% Volitional vs Automatic movement

What are the control requirements - high level(perception) $\rightarrow$ low level\cite{Tucker2015}


%---------------------------------------------%
\section{Machine Learning}
Introduction to ML section - background of ML

\subsection{Development}
History of ML

\subsection{Recurrent Neural Networks}
RNN
GRU
LSTM
CNN-LSTM

\subsection{Transformer Networks}
Brief description of them


%---------------------------------------------%
\section{Related Works} - % The Lit Review Section
How have people gone about solving the problems described above - what are the limitation and identify research gaps

Recap problems

\subsection{Sensors}
What sensors have people used

\subsection{Heuristics}
What heuristics have people used

\subsection{ML} % This will be the bulk of this section
ML Methods have people used


Link all of this back to the research question/aims and objectives of the thesis

